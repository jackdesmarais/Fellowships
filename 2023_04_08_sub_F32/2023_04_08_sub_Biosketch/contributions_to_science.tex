%!TEX output_directory = build
%!TEX copy_output_on_build = true

\documentclass{article}

\raggedright

\usepackage[style=authoryear, backend=biber, maxnames=999]{biblatex}
\addbibresource{/Users/jackdesmarais/Documents/Kinney_lab/Fellowships/2023_04_08_sub_F32/2023_04_08_sub_Biosketch/paperpile2.bib} %Imports bibliography file

\newcounter{namesnotimportant}
\newtoggle{ellipsis}

\makeatletter
\newbibmacro*{name:etal:delim}[1]{%
  \ifnumgreater{\value{listcount}}{\value{liststart}}
    {\ifboolexpr{
       test {\ifnumless{\value{listcount}}{\value{liststop}}}
       or
       test \ifmorenames
       or test {\ifnumcomp{\value{namesnotimportant}}{>}{0}}
     }
       {\printdelim{multinamedelim}}
       {\lbx@finalnamedelim{#1}}}
    {}}
\makeatother

\DeclareNameFormat{given-family-etal}{%
  \letbibmacro{name:delim}{name:etal:delim}%
  \ifnumcomp{\value{listcount}}{=}{1}
    {\setcounter{namesnotimportant}{0}%
     \global\toggletrue{ellipsis}}
    {}%
  \ifboolexpr{test {\ifnumcomp{\value{listcount}}{=}{1}}
              or test {\ifnumcomp{\value{listtotal}}{=}{2}}}
    {\ifgiveninits
      {\usebibmacro{name:given-family}
         {\namepartfamily}
         {\namepartgiveni}
         {\namepartprefix}
         {\namepartsuffix}}
      {\usebibmacro{name:given-family}
         {\namepartfamily}
         {\namepartgiven}
         {\namepartprefix}
         {\namepartsuffix}}}%
    {\ifboolexpr{test {\iffieldequalstr{hash}{d044631f3f9582c1e40e11481a6d06e9}}
            or 
            test {\iffieldequalstr{hash}{d36f79db286d144cbc39cc5559a702d3}}
            or 
            test {\iffieldequalstr{hash}{02913108490bef3778a320768a68b0d9}}}%% <----- put the correct hash here
      {\global\toggletrue{ellipsis}%
       \ifgiveninits
        {\usebibmacro{name:given-family}
           {\namepartfamily}
           {\namepartgiveni}
           {\namepartprefix}
           {\namepartsuffix}}
        {\usebibmacro{name:given-family}
           {\namepartfamily}
           {\namepartgiven}
           {\namepartprefix}
           {\namepartsuffix}}}%
      {\stepcounter{namesnotimportant}%
       \iftoggle{ellipsis}
         {\addcomma\space\textellipsis\global\togglefalse{ellipsis}\isdot}
         {}}}%
  \ifboolexpr{
    test {\ifnumequal{\value{listcount}}{\value{liststop}}}
    and
    (test \ifmorenames
     or test {\ifnumcomp{\value{namesnotimportant}}{>}{0}})
  }
    {\andothersdelim\bibstring{andothers}}
    {}}

\DeclareNameAlias{sortname}{given-family-etal}
\DeclareNameAlias{author}{given-family-etal}
\DeclareNameAlias{editor}{given-family-etal}
\DeclareNameAlias{translator}{given-family-etal}


\usepackage{soul}

\usepackage[letterpaper,
            bindingoffset=0.2in,
            left=1in,
            right=1in,
            top=1in,
            bottom=1in,
            footskip=.25in]{geometry}

\usepackage{titlesec} 
\titleformat{\section}[runin]
  {\normalfont\normalsize\bfseries}{}{0pt}{}
\titlespacing{\section}{0pt}{-\parskip}{1mm}

\renewcommand{\thesubsection}{\arabic{subsection}}
\titleformat{\subsection}[runin]
  {\normalfont\normalsize\bfseries}{Contribution \thesubsection: }{0pt}{}
\titlespacing{\subsection}{0pt}{-\parskip}{1mm}

\titleformat{\subsubsection}[runin]
  {\normalfont\normalsize\itshape}{\thesubsubsection: }{0pt}{\ul}
\titlespacing{\subsubsection}{0pt}{-\parskip}{1mm}

\begin{document}


\section*{Contributions to Science:}
\newrefsection
\subsection{Discovery of a new type of carbon pump that drives CO$_2$ concentration in bacteria:}
\subsubsection{Historical background}
Rubisco is the enzyme responsible for fixing the vast majority of all the CO$_2$ that enters the biosphere each year and is essential for all plants, algae, and most autotrophic bacteria. 
However, Rubisco evolved before the great oxygenation event and is competitively inhibited by the presence of O$_2$.
This is a problem that all autotrophs that use rubisco must overcome to survive in the the modern atmosphere, with its 20\% O$_2$ and only 0.04\% CO$_2$.
Many autotrophic bacteria including a wide range of chemolithoautotrophs and most cyanobacteria overcome this issue using an $\alpha$-carboxysome based CO$_2$ concentrating mechanism (CCM).
These CCMs work by creating a compartment inside the bacteria where the concentration of CO$_2$ is raised high enough to saturate rubisco with CO$_2$ and out-compete O$_2$.
While our theoretical understanding and previous experimental work suggested that an inorganic carbon (C$_i$) transporter was absolutely required for the functioning of the CCM, no such transporter was known in the model chemolithoautotroph \textit{H. neapolitanus}.
%
\subsubsection{Central finding}
I performed a genome-wide genetic screen for CCM components in \textit{H. neapolitanus}. 
I identified the essential components of the CCM in \textit{H. neapolitanus}.
I cloned putative C$_i$ transporters, called DABs, then confirmed their activity with reporter strain and C$_i$ uptake assays in \textit{E. coli}. 
I showed that data was consistent with these transporters acting not as direct transporters but as a new family of energy coupled carbonic anhydrase (CA) enzymes.
These CAs concentrate C$_i$ by converting membrane permeable CO$_2$ into membrane impermeable HCO$_{3}^{-}$ causing a net flow of C$_i$ into the cell.
I further showed that similar operons in the human pathogens \textit{V. Cholera} and \textit{B. anthracis} had the same function.
%
\subsubsection{Influence/Application}
This work was instrumental in ongoing study of the CCM including a successful effort to reconstitute a functional $\alpha$-carboxysome CCM in \textit{E. coli}, and work on CCM evolution that I will discuss in contribution \ref{CCM_evo}. 
Since this work was published, it was shown that a similar operon in \textit{S. aureus} has the same function and is essential for growth in atmospheric CO$_2$ concentrations. 
Considering the activity of energy coupled CA transporters has been shown to be necessary to understand the carbon isotope fractionation data in very old rock strata.
There has been interest in investigating the potential applications of DABs in engineering crop plants and autotrophic bio-fuel production hosts (like \textit{C. necator}) for increased yield.
%
\subsubsection{My role}
I was the primary author on this work.
I conceived and designed the experiments, performed the genetic screen, analyzed the sequencing data, did cloning, performed the biochemistry experiments, and performed the reporter strain experiments.
%
\nocite{Desmarais2019-yc}
\printbibliography[heading=none]

\newrefsection
\subsection{Characterization of potential evolutionary paths for developing carbon dioxide concentrating mechanisms:} \label{CCM_evo}
\subsubsection{Historical background}
The $\alpha$-carboxysome based CO$_2$ concentrating mechanism (CCM) required several major evolutionary steps to evolve.
These were acquiring uncoupled CA activity, gaining C$_i$ transport, and encapsulating CA and rubisco in an $\alpha$-carboxysome.
However, all of these components are need for the effect of the CCM and removing even one of these components is lethal in modern autotrophs.
Since there is no apparent fitness benefit for a partial system, it is not clear how the system could have evolved.
Geochemical evidence suggests that the atmosphere was very different when the CCM first evolved, with much higher levels of CO$_2$ and much lower levels of O$_2$ and a mixture of biological and geochemical processes has slowly changed the atmospheric composition to the current mixture.
This information lead us to the hypothesis that evolutionary intermediates of the CCM may have provided fitness benefits at intermediate atmospheric compositions.
%
\subsubsection{Central finding}
We compared the effect of CCM gene knockouts in \textit{H. neapolitanus} across different intermediate CO$_2$ concentrations.
We also compared the growth phenotypes of partial CCM constructs in CO$_2$ concentration dependent strains of \textit{E. coli} and \textit{C. necator} that we constructed.
We found that introduction of an uncoupled CA (non-transporter) or a coupled CA (C$_i$ transporter) provided a fitness benefit at intermediate CO$_2$ concentrations.
We also found that while combining either of these activities with the $\alpha$-carboxysome on their own provided no benefit, combining them with each other provided benefits in some situations. 
This was unexpected, because including a coupled and an uncoupled CA in the same cell without encapsulation is expected to produce a futile cycle.
Our mathematical modeling suggests that at intermediate CO$_2$ concentrations and high growth rates the cell can become limited by both CO$_2$ and HCO$_3$ and C$_i$ consumption is high enough that cycling is actually beneficial for providing both C$_i$ species for growth.
This revealed a possible path to evolving a CCM through only fitness positive steps by acquiring first either a coupled or uncoupled CA, then acquiring the other type of CA as CO$_2$ fall further, and finally evolving an $\alpha$-carboxysome as CO$_2$ concentrations approach modern levels.
%
\subsubsection{Influence/Application}
This work provides insight into the evolution of the CCM and into possible life strategies of modern organisms living in hign CO$_2$ environments.
These strategies might also be useful for improving the growth of industrial autotrophs at intermediate CO$_2$ concentrations.
Further, showing the functional expression of the DAB in \textit{C. necator} offers the potential for using the DAB to improve bioplastics production.
%
\subsubsection{My role}
My role in this project was to measure the effect of all CCM gene knockouts on the growth of \textit{H. neapolitanus} across a panel of intermediate CO$_2$ concentrations to establish which components of it's CCM are needed for growth at intermediate CO$_2$ concentrations.
%
\nocite{Flamholz2022-yo}
\printbibliography[heading=none]

\newrefsection
\subsection{Development of nuclease chain reaction signal amplification for viral diagnostics:}
\subsubsection{Historical background}
%
\subsubsection{Central finding}
%
\subsubsection{Influence/Application}
%
\subsubsection{My role}
%
\nocite{Liu2021-pu}
\printbibliography[heading=none]

% \newrefsection
\subsection{Application of general epistasis techniques to protein design:}
\subsubsection{Historical background}
%
\subsubsection{Central finding}
%
\subsubsection{Influence/Application}
%
\subsubsection{My role}
%
% \printbibliography[heading=none]

\newrefsection
\subsection{Discovery of CasX:}
\subsubsection{Historical background}
%
\subsubsection{Central finding}
%
\subsubsection{Influence/Application}
%
\subsubsection{My role}
%
\nocite{Liu2019-nk}
\printbibliography[heading=none]

\newrefsection
\subsection{Improved production of chemicals in E. coli through nitrogen limitation:}
\subsubsection{Historical background}
%
\subsubsection{Central finding}
%
\subsubsection{Influence/Application}
%
\subsubsection{My role}
%
\nocite{Chubukov2017-uu}
\printbibliography[heading=none]


\end{document}

