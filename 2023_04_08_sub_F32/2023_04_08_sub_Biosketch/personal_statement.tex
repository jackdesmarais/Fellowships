%!TEX output_directory = build
%!TEX copy_output_on_build = true

\documentclass{article}

\raggedright

\usepackage[style=numeric-comp, backend=biber,url=false,eprint=false]{biblatex}
\addbibresource{./mypapers.bib} %Imports bibliography file

\usepackage{soul}

\usepackage[letterpaper,
            bindingoffset=0.2in,
            left=1in,
            right=1in,
            top=1in,
            bottom=1in,
            footskip=.25in]{geometry}

\usepackage{titlesec} 
\titleformat{\section}[runin]
  {\normalfont\normalsize\bfseries}{}{0pt}{}
\titlespacing{\section}{0pt}{-\parskip}{1mm}

\renewcommand{\thesubsection}{\arabic{subsection}}
\titleformat{\subsection}[runin]
  {\normalfont\normalsize\bfseries}{Contribution \thesubsection: }{0pt}{}
\titlespacing{\subsection}{0pt}{-\parskip}{1mm}

\titleformat{\subsubsection}[runin]
  {\normalfont\normalsize\itshape}{\thesubsubsection: }{0pt}{\ul}
\titlespacing{\subsubsection}{0pt}{-\parskip}{1mm}




\begin{document}
\section*{Personal statement:}
I am a perfect fit for this F32 fellowship project because it leverages my past experience and will position me well for my ideal next steps.
My ideal career is to lead a lab that focuses on using a mixture of massively parallel assays and computational modeling to understand  biological questions.
I value the ability to pursue fundamental questions but also want to see the answers to those questions make a difference in the lives of people. 
Because of this, I aim to work at a university or research institute where I can study both the fundamental and the applied sides of problems. 
I am broadly interested in asking questions about how information encoded in the sequence of biological molecules controls the phenotypes we observe and in working on applications in human health or climate. 
Through these topics, I hope to be able to both contribute to our understanding of biology and to apply that knowledge to real problems.
%TRANSITIONAL SENTENCE!!!!
My proposed project leverages massively parallel splicing assays to provide an unbiased look at sequence elements that are important for establishing mutual exclusivity between exons 9 and 10 in pyruvate kinase M.
I will follow up this work with both targeted experiments and modeling to elucidate the mechanisms of mutual exclusivity.
My previous research has given me the skills I will need for this project.
During my Ph.D. I performed a massively parallel growth assays to identify genes involved in the CO$_2$ concentrating mechanism in a chemoautotroph and measure their phenotypes as a function of CO$_2$ concentration.
I followed up these experiments with detailed mechanistic experiments including reporter strain growth experiments, \supercite{Desmarais2019-yc,Flamholz2022-yo}
I also used a mixture of massively parallel growth assays and modeling to map the fitness landscape of dihydrofolate reductase and design new functional variants in work that is in prep.
I further demonstrated my ability at biochemical identifying the zinc binding behavior of the newly discovered CasX.\supercite{Liu2019-nk} 
Finally, I applied kinetic modeling to design signal amplification strategies for cas13-based detection of RNA viruses.\supercite{Liu2021-pu}
These experiences prepared me well to use massively parallel assays, focused genetic and biochemical experiments, and modeling to characterize splicing regulation. 
This project will help me achieve my scientific goals by broadening my experience.
My previous work has focused on applying massively parallel techniques and modeling to the study of microbiology and protein engineering.
This project will allow me to explore applications in RNA biology, in the human-cell context, and with long read sequencing read outs.
Further, this experience will allow me to grow and strengthen my experience with computational modeling by enabling me to work in the Kinney lab and in the Cold Spring Harbor Laboratory Simons Center for Quantitative biology, where I can work with world class experts on modeling of biological systems.




% I am uniquely qualified for this F32 application because it leverages my expertise in massively parallel assays, biochemistry and, quantitative modeling.
% \subsubsection*{massively parallel assays}
% In graduate school, I performed a massively parallel growth assay to measure quantitative phenotypes for $\sim$100,000 non-lethal (at 10\% CO$_2$) gene knockouts in the chemolithoautotroph \textit{H. neapolitanus} as a function of CO$_2$ concentration.\supercite{Desmarais2019-yc,Flamholz2022-yo}
% After this, I used massively parallel growth assays to measure enzyme kinetics \textit{in vivo} by calibrating growth rates of individual strains in mixed cultures kept at mid log phase against measured enzyme kinetic parameters in work I hope will be published soon.
% Through this experience, I learned how to design, perform, optimize, and analyze massively parallel assays to gather high quality quantitative data that varies as a function of external variables.
% I also have experience with building mutant libraries using programed transposon mutagenesis,\supercite{Desmarais2019-yc} error prone PCR, and mutated oligo libraries as well as preparing libraries for Illumina sequencing.\supercite{Desmarais2019-yc}
% This experience will aid me in designing and performing the massively parallel splicing assays I propose in this work.
% \subsubsection*{biochemistry} %I think this may be the wrong term here, I want to also include my uptake/reporter assays... think this over
% I have experience using biochemisty(?) techniques to follow up on interesting hits from screens. 
% This includes techniques as diverse as reporter strain growth assays, C$^14$ uptake assays, protein co-purification purification, enzyme assays, site specific mutagenesis, and x-ray fluorescence spectroscopy.\supercite{Desmarais2019-yc,Liu2019-nk} 
% This experience will prepare me for the work I have proposed here to investigate the mechanisms of interesting hits from the screen.
% \subsubsection*{quantitative modeling}
% I have experience writing data analysis pipelines for Illumina sequencing data and for quantitative data more broadly.\supercite{Desmarais2019-yc,Liu2021-pu,Flamholz2022-yo}
% Further I have demonstrated expertise performing kinetic modeling of complicated multi-step reactions.\supercite{Liu2021-pu}
% Finally, in recent as yet unpublished work, I have developed skill at modeling with machine learning models including neural nets, general epistasis models, and potts models.
% These skills will prepare me well for building quantitative models of splicing informed by both phylogeny and my massively parallel assay data as I propose here.



\printbibliography[heading=none]


\end{document}