%!TEX program = lualatex
%!TEX output_directory = build
%!TEX copy_output_on_build = true

\documentclass[11pt]{article}

\usepackage[
backend=biber,
url=false,
eprint=false, 
dashed=false,
style=authoryear,
sorting=none,
maxnames=999
]{biblatex}

\addbibresource{paperpile.bib} %Imports bibliography file

% \usepackage[style=authoryear, 
%             backend=biber,
%             url=false,
%             eprint=false, 
%             dashed=false,
%             maxnames=999]{biblatex}
% \addbibresource{mypapers.bib} %Imports bibliography file

% \newcounter{namesnotimportant}
% \newtoggle{ellipsis}

% \makeatletter
% \newbibmacro*{name:etal:delim}[1]{%
%   \ifnumgreater{\value{listcount}}{\value{liststart}}
%     {\ifboolexpr{
%        test {\ifnumless{\value{listcount}}{\value{liststop}}}
%        or
%        test \ifmorenames
%        or test {\ifnumcomp{\value{namesnotimportant}}{>}{0}}
%      }
%        {\printdelim{multinamedelim}}
%        {\lbx@finalnamedelim{#1}}}
%     {}}
% \makeatother

% \DeclareNameFormat{given-family-etal}{%
%   \letbibmacro{name:delim}{name:etal:delim}%
%   \ifnumcomp{\value{listcount}}{=}{1}
%     {\setcounter{namesnotimportant}{0}%
%      \global\toggletrue{ellipsis}}
%     {}%
%   \ifboolexpr{test {\ifnumcomp{\value{listcount}}{=}{1}}
%               or test {\ifnumcomp{\value{listtotal}}{=}{2}}}
%     {\ifgiveninits
%       {\usebibmacro{name:given-family}
%          {\namepartfamily}
%          {\namepartgiveni}
%          {\namepartprefix}
%          {\namepartsuffix}}
%       {\usebibmacro{name:given-family}
%          {\namepartfamily}
%          {\namepartgiven}
%          {\namepartprefix}
%          {\namepartsuffix}}}%
%     {\ifboolexpr{test {\iffieldequalstr{hash}{d044631f3f9582c1e40e11481a6d06e9}}
%             or 
%             test {\iffieldequalstr{hash}{d36f79db286d144cbc39cc5559a702d3}}
%             or 
%             test {\iffieldequalstr{hash}{02913108490bef3778a320768a68b0d9}}}%% <----- put the correct hash here
%       {\global\toggletrue{ellipsis}%
%        \ifgiveninits
%         {\usebibmacro{name:given-family}
%            {\namepartfamily}
%            {\namepartgiveni}
%            {\namepartprefix}
%            {\namepartsuffix}}
%         {\usebibmacro{name:given-family}
%            {\namepartfamily}
%            {\namepartgiven}
%            {\namepartprefix}
%            {\namepartsuffix}}}%
%       {\stepcounter{namesnotimportant}%
%        \iftoggle{ellipsis}
%          {\addcomma\space\textellipsis\global\togglefalse{ellipsis}\isdot}
%          {}}}%
%   \ifboolexpr{
%     test {\ifnumequal{\value{listcount}}{\value{liststop}}}
%     and
%     (test \ifmorenames
%      or test {\ifnumcomp{\value{namesnotimportant}}{>}{0}})
%   }
%     {\andothersdelim\bibstring{andothers}}
%     {}}

% \DeclareNameAlias{sortname}{given-family-etal}
% \DeclareNameAlias{author}{given-family-etal}
% \DeclareNameAlias{editor}{given-family-etal}
% \DeclareNameAlias{translator}{given-family-etal}

\raggedright

\usepackage[no-math]{fontspec}
\setmainfont{Arial}
\newfontfamily\latofont{Arial}

\usepackage{soul}

\usepackage[letterpaper,
            bindingoffset=0.2in,
            left=0.5in,
            right=0.5in,
            top=0.5in,
            bottom=0.5in,
            footskip=.25in]{geometry}

\usepackage{titlesec} 
\titleformat{\section}[hang]
  {\normalfont\normalsize\bfseries}{}{1ex}{}
\titlespacing{\section}{0pt}{-\parskip}{1mm}

\titleformat{\subsection}[runin]
  {\normalfont\normalsize}{}{1mm}{\ul}
\titlespacing{\subsection}{0pt}{11pt}{1mm}

\renewcommand{\thesubsubsection}{\arabic{subsubsection}}
\titleformat{\subsubsection}[runin]
  {\normalfont\normalsize\bfseries}{\thesubsubsection}{1mm}{}
\titlespacing{\subsubsection}{0pt}{11pt}{1mm}

\titleformat{\paragraph}[runin]
  {\normalfont\normalsize\itshape}{\theparagraph}{1pt}{\ul}
\titlespacing{\paragraph}{0pt}{-\parskip}{1mm}

\begin{document}

\section*{Doctoral Dissertation and Research Experience}
%
\subsection{Predoctoral research}
\subsubsection{Neitz lab}
\begin{refsection}
\paragraph{Mentor}
Dr. Jay Neitz
%
\paragraph{Role}
Summer intern, Neitz lab, University of Washington medical school, 2011-2012
%
\paragraph{Description}
My first research experiences were summer internships in the Neitz lab during high school. 
I worked with postdoctoral fellows and graduate students in the lab to validate a mouse model for introduction of new photo-receptors in adult animals to study color vision neural circuitry function. 
As part of this project I worked on a color discrimination test for mice, took confocal images of mouse retinas, and measured electroretinograms of mice after pilot experiments.
I presented this work at two local competitions and a regional competition.
%
\nocite{Desmarais2012-ki,Desmarais2012-oz,Desmarais2012-nx}
\printbibliography[heading=none]
\end{refsection}
%
\subsubsection{STEM innovation program}
\paragraph{Mentor}
Drs. Noah Graham, Frank Swenton, Jeremy Ward
%
\paragraph{Role}
Researcher, STEM innovation program, Middlebury College, 2013
%
\paragraph{Description}
During my freshman year at college, I applied to and joined the STEM innovation program, where a group of 9 undergraduates designed and executed a synthetic biology project. 
We set out to create a biosensor for detecting aromatic hydrocarbons in water samples, focusing on benzene, toluene, ethylbenzene and xylene (BTEX) as they have been detected in groundwater near fracking wells. 
We designed and built constructs using a transcriptional regulator from \textit{P. putida}, tested lyophilization/rehydration protocols, and built and tested a portable fluorescence reader for testing samples. 
%
\subsubsection{Gibson lab}
%
\paragraph{Mentor}
Drs. Matthew C. Gibson, Aissam Ikmi
%
\paragraph{Role}
Stowers Summer Scholar, Gibson lab, Stowers Institute for Medical Research, 2014
%
\paragraph{Description}
During the summer after my sophomore year, I was a Stowers summer scholar in Matthew Gibson's lab.
I worked with the postdoc Dr. Aissam Ikmi, studying how the size on the embryo affects the early development of \textit{N. vectensis} sea anemones.
\textit{N. vectensis} grows from an egg into a polyp with 4 tentacles and then begins to eat and grow.
At the 4 cell stage, all 4 cells still retain their ability to produce a viable polyp.
By subdividing embryos and observing their development, we hoped to observe the effects of size on developmental processes like tentacle patterning.
%
\paragraph{Results}
Reducing embryo size also reduced polyp size, reducing length more than width, and tentacle number.
Mesentery number changed with tentacle number maintaining a ratio of two mesenteries for each tentacle.
Regardless of initial size, polyps grew to similar sizes before developing the first pair of additional tentacles.
All together, this data suggested that tentacles are patterned in a size dependent manner.
%
\subsubsection{Keasling lab}
\begin{refsection}
%
\paragraph{Mentor}
Drs. Jay Keasling, Victor Chubukov
%
\paragraph{Role}
Amgen Scholar, Keasling lab, Joint Bioenergy Institute, University of California, Berkeley, 2015
%
\paragraph{Description}
After my junior year of college, I joined the Keasling lab for the summer as an Amgen Scholar.
Under the mentorship of postdoc Dr. Victor Chubukov, I worked on developing chassis strains of \textit{E. coli} that could be used to improve yields in bio-production of fuels and chemicals.
By developing chassis strains, we hoped to provide broadly applicable methods that applied to a variety of different target chemicals.
One major issue encountered by metabolic engineers in producing chemicals is shunting carbon and energy towards growth not production. 
In order to avoid this growth and production can be separated by placing the cells in growth limiting conditions while inducing the production pathway.
However, many chassis strains will go dormant under these conditions.
We tested the hypothesis that increasing glucose uptake during nitrogen starvation would increase the amount of carbon and energy that could be directed to production.
%
\paragraph{Results}
In nitrogen starvation, $\alpha$-ketoglutarate levels rise, inhibiting the enzyme PtsI.
This blocks glucose phosphorylation and therefore uptake.
We trialed 3 methods to overcome this regulation.
The first was over-expressing PtsI.
The second was a PtsI-PtsP chimera that was hypothesized to avoid inhibition.
The third was GalP and Glk a permease and kinase that take up glucose through an orthogonal pathway.
We found that all of these strategies increase glucose uptake during nitrogen limitation, but PtsI over-expression is more effective than chimera over-expression, and Galp/Glk over-expression can cause cell death.
The PtsI over-expression strain consumed 4x more glucose than WT during nitrogen starvation despite optical density staying constant and a lack of fermentation byproducts being secreted.
This suggests that the glucose was converted all the way to CO$_2$ by the TCA cycle.
We then tested if this strategy improved yield in a fatty alcohol production experiment.
We saw that while nitrogen staring the cells increased carbon use efficiency, increasing their metabolic activity with PtsI did not improve yield.
We published these finding in NPJ systems and synthetic biology.
%
\nocite{Chubukov2017-uu}
\printbibliography[heading=none]
\end{refsection}
%
\subsubsection{Ward lab}
%
\paragraph{Mentor}
Dr. Jeremy Ward
%
\paragraph{Role}
Researcher, Ward lab, Middlebury College, 2014-2016
%
\paragraph{Description}
%
\paragraph{Results}
%
\subsubsection{Connection to fellowship}
These experiences helped me to decide to pursue a career as a scientist and launched my scientific career. 
These early projects exposed me to a wide variety of research areas and helped to guide me towards my eventual area of focus.
These early research projects also gave me a wide breadth of experience with different techniques and organisms.
Finally, they allowed me to build my skills planning and managing research projects, expecially through my time running independent projects in the Ward lab and the STEM Innovation Project.
%
\subsection{Doctoral research}
\subsubsection{Savage lab - Studies of the CO$_2$ concentrating mechanism}
\begin{refsection}
%
\paragraph{Mentor}
Dr. David Savage
%
\paragraph{Role}
Graduate Student Researcher, Savage lab, University of California, Berkeley, 2016-2022
%
\paragraph{Description}
%
\paragraph{Results}
%
\paragraph{Connection to fellowship}
%
\nocite{Desmarais2019-dc,Flamholz2022-yo}
\printbibliography[heading=none]
\end{refsection}
%
\subsubsection{Savage lab - Protein engineering}
%
\paragraph{Mentor}
Dr. David Savage
%
\paragraph{Role}
Graduate Student Researcher, Savage lab, University of California, Berkeley, 2016-2022
%
\paragraph{Description}
%
\paragraph{Results}
%
\paragraph{Connection to fellowship}
%
\subsubsection{Savage lab - CRISPR tool development}
\begin{refsection}
%
\paragraph{Mentor}
Dr. David Savage
%
\paragraph{Role}
Graduate Student Researcher, Savage lab, University of California, Berkeley, 2016-2022
%
\paragraph{Description}
%
\paragraph{Results}
%
\paragraph{Connection to fellowship}
%
\nocite{Liu2019-nk,Liu2021-pu}
\printbibliography[heading=none]
\end{refsection}
%
\subsection{postdoctoral research}
\subsubsection{Kinney lab}
%
\paragraph{Mentor}
Dr. Justin Kinney
%
\paragraph{Role}
Computational Postdoctoral Fellow, Kinney Lab, Cold Spring Harbor Laboratory, 2023
%
\paragraph{Description}
%
\paragraph{Results}
%
\paragraph{Connection to fellowship}
%
\section*{Goals for the Fellowship and Training}
%
\subsection{Overall training goals}
%
\subsection{Skills to be enhanced}
%
\subsection{Preparation for career plans}
%
\section*{Activities Planned Under This Award}





\end{document}
